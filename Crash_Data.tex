% Options for packages loaded elsewhere
\PassOptionsToPackage{unicode}{hyperref}
\PassOptionsToPackage{hyphens}{url}
%
\documentclass[
]{article}
\usepackage{amsmath,amssymb}
\usepackage{lmodern}
\usepackage{iftex}
\ifPDFTeX
  \usepackage[T1]{fontenc}
  \usepackage[utf8]{inputenc}
  \usepackage{textcomp} % provide euro and other symbols
\else % if luatex or xetex
  \usepackage{unicode-math}
  \defaultfontfeatures{Scale=MatchLowercase}
  \defaultfontfeatures[\rmfamily]{Ligatures=TeX,Scale=1}
\fi
% Use upquote if available, for straight quotes in verbatim environments
\IfFileExists{upquote.sty}{\usepackage{upquote}}{}
\IfFileExists{microtype.sty}{% use microtype if available
  \usepackage[]{microtype}
  \UseMicrotypeSet[protrusion]{basicmath} % disable protrusion for tt fonts
}{}
\makeatletter
\@ifundefined{KOMAClassName}{% if non-KOMA class
  \IfFileExists{parskip.sty}{%
    \usepackage{parskip}
  }{% else
    \setlength{\parindent}{0pt}
    \setlength{\parskip}{6pt plus 2pt minus 1pt}}
}{% if KOMA class
  \KOMAoptions{parskip=half}}
\makeatother
\usepackage{xcolor}
\usepackage[margin=1in]{geometry}
\usepackage{color}
\usepackage{fancyvrb}
\newcommand{\VerbBar}{|}
\newcommand{\VERB}{\Verb[commandchars=\\\{\}]}
\DefineVerbatimEnvironment{Highlighting}{Verbatim}{commandchars=\\\{\}}
% Add ',fontsize=\small' for more characters per line
\usepackage{framed}
\definecolor{shadecolor}{RGB}{248,248,248}
\newenvironment{Shaded}{\begin{snugshade}}{\end{snugshade}}
\newcommand{\AlertTok}[1]{\textcolor[rgb]{0.94,0.16,0.16}{#1}}
\newcommand{\AnnotationTok}[1]{\textcolor[rgb]{0.56,0.35,0.01}{\textbf{\textit{#1}}}}
\newcommand{\AttributeTok}[1]{\textcolor[rgb]{0.77,0.63,0.00}{#1}}
\newcommand{\BaseNTok}[1]{\textcolor[rgb]{0.00,0.00,0.81}{#1}}
\newcommand{\BuiltInTok}[1]{#1}
\newcommand{\CharTok}[1]{\textcolor[rgb]{0.31,0.60,0.02}{#1}}
\newcommand{\CommentTok}[1]{\textcolor[rgb]{0.56,0.35,0.01}{\textit{#1}}}
\newcommand{\CommentVarTok}[1]{\textcolor[rgb]{0.56,0.35,0.01}{\textbf{\textit{#1}}}}
\newcommand{\ConstantTok}[1]{\textcolor[rgb]{0.00,0.00,0.00}{#1}}
\newcommand{\ControlFlowTok}[1]{\textcolor[rgb]{0.13,0.29,0.53}{\textbf{#1}}}
\newcommand{\DataTypeTok}[1]{\textcolor[rgb]{0.13,0.29,0.53}{#1}}
\newcommand{\DecValTok}[1]{\textcolor[rgb]{0.00,0.00,0.81}{#1}}
\newcommand{\DocumentationTok}[1]{\textcolor[rgb]{0.56,0.35,0.01}{\textbf{\textit{#1}}}}
\newcommand{\ErrorTok}[1]{\textcolor[rgb]{0.64,0.00,0.00}{\textbf{#1}}}
\newcommand{\ExtensionTok}[1]{#1}
\newcommand{\FloatTok}[1]{\textcolor[rgb]{0.00,0.00,0.81}{#1}}
\newcommand{\FunctionTok}[1]{\textcolor[rgb]{0.00,0.00,0.00}{#1}}
\newcommand{\ImportTok}[1]{#1}
\newcommand{\InformationTok}[1]{\textcolor[rgb]{0.56,0.35,0.01}{\textbf{\textit{#1}}}}
\newcommand{\KeywordTok}[1]{\textcolor[rgb]{0.13,0.29,0.53}{\textbf{#1}}}
\newcommand{\NormalTok}[1]{#1}
\newcommand{\OperatorTok}[1]{\textcolor[rgb]{0.81,0.36,0.00}{\textbf{#1}}}
\newcommand{\OtherTok}[1]{\textcolor[rgb]{0.56,0.35,0.01}{#1}}
\newcommand{\PreprocessorTok}[1]{\textcolor[rgb]{0.56,0.35,0.01}{\textit{#1}}}
\newcommand{\RegionMarkerTok}[1]{#1}
\newcommand{\SpecialCharTok}[1]{\textcolor[rgb]{0.00,0.00,0.00}{#1}}
\newcommand{\SpecialStringTok}[1]{\textcolor[rgb]{0.31,0.60,0.02}{#1}}
\newcommand{\StringTok}[1]{\textcolor[rgb]{0.31,0.60,0.02}{#1}}
\newcommand{\VariableTok}[1]{\textcolor[rgb]{0.00,0.00,0.00}{#1}}
\newcommand{\VerbatimStringTok}[1]{\textcolor[rgb]{0.31,0.60,0.02}{#1}}
\newcommand{\WarningTok}[1]{\textcolor[rgb]{0.56,0.35,0.01}{\textbf{\textit{#1}}}}
\usepackage{graphicx}
\makeatletter
\def\maxwidth{\ifdim\Gin@nat@width>\linewidth\linewidth\else\Gin@nat@width\fi}
\def\maxheight{\ifdim\Gin@nat@height>\textheight\textheight\else\Gin@nat@height\fi}
\makeatother
% Scale images if necessary, so that they will not overflow the page
% margins by default, and it is still possible to overwrite the defaults
% using explicit options in \includegraphics[width, height, ...]{}
\setkeys{Gin}{width=\maxwidth,height=\maxheight,keepaspectratio}
% Set default figure placement to htbp
\makeatletter
\def\fps@figure{htbp}
\makeatother
\setlength{\emergencystretch}{3em} % prevent overfull lines
\providecommand{\tightlist}{%
  \setlength{\itemsep}{0pt}\setlength{\parskip}{0pt}}
\setcounter{secnumdepth}{-\maxdimen} % remove section numbering
\ifLuaTeX
  \usepackage{selnolig}  % disable illegal ligatures
\fi
\IfFileExists{bookmark.sty}{\usepackage{bookmark}}{\usepackage{hyperref}}
\IfFileExists{xurl.sty}{\usepackage{xurl}}{} % add URL line breaks if available
\urlstyle{same} % disable monospaced font for URLs
\hypersetup{
  pdftitle={CrashData Analysis},
  pdfauthor={Student},
  hidelinks,
  pdfcreator={LaTeX via pandoc}}

\title{CrashData Analysis}
\author{Student}
\date{2023-04-12}

\begin{document}
\maketitle

\hypertarget{load-the-required-libraries}{%
\subsubsection{Load the required
libraries}\label{load-the-required-libraries}}

\begin{Shaded}
\begin{Highlighting}[]
\FunctionTok{library}\NormalTok{(ggplot2)}
\FunctionTok{library}\NormalTok{(dplyr)}
\end{Highlighting}
\end{Shaded}

\begin{verbatim}
## 
## Attaching package: 'dplyr'
\end{verbatim}

\begin{verbatim}
## The following objects are masked from 'package:stats':
## 
##     filter, lag
\end{verbatim}

\begin{verbatim}
## The following objects are masked from 'package:base':
## 
##     intersect, setdiff, setequal, union
\end{verbatim}

\begin{Shaded}
\begin{Highlighting}[]
\FunctionTok{library}\NormalTok{(tidyr)}
\end{Highlighting}
\end{Shaded}

\hypertarget{read-the-csv-file-and-store-it-in-a-data-frame}{%
\subsubsection{Read the CSV file and store it in a data
frame}\label{read-the-csv-file-and-store-it-in-a-data-frame}}

We can see top elements of the data.

\begin{Shaded}
\begin{Highlighting}[]
\CommentTok{\# Read the CSV file and store it in a data frame}
\NormalTok{crash\_data }\OtherTok{\textless{}{-}} \FunctionTok{read.csv}\NormalTok{(}\StringTok{"crash\_data.csv"}\NormalTok{)}
\FunctionTok{head}\NormalTok{(crash\_data)}
\end{Highlighting}
\end{Shaded}

\begin{verbatim}
##   Crash_Year Crash_Month  County           Crash_Type    Crash_Drug_Use
## 1       2013     January   Barry              Head-On No Drugs Involved
## 2       2013     January   Barry             Rear-End No Drugs Involved
## 3       2013     January     Bay Single Motor Vehicle No Drugs Involved
## 4       2013     January Berrien             Rear-End No Drugs Involved
## 5       2013     January  Branch                Angle No Drugs Involved
## 6       2013     January Calhoun  Head-On - Left Turn No Drugs Involved
##         Crash_Drinking Crash_Hit_and_Run Speed_Limit_at_Crash_Site
## 1 No Drinking Involved   Not Hit-and-Run                        55
## 2 No Drinking Involved   Not Hit-and-Run                        55
## 3 No Drinking Involved   Not Hit-and-Run                        55
## 4 No Drinking Involved   Not Hit-and-Run                        70
## 5 No Drinking Involved   Not Hit-and-Run                        55
## 6    Drinking Involved   Not Hit-and-Run                        45
##   Total_Motor_Vehicles Crash
## 1                    2     0
## 2                    2     0
## 3                    1     0
## 4                    2     0
## 5                    2     0
## 6                    2     0
\end{verbatim}

\hypertarget{visualisation}{%
\section{Visualisation}\label{visualisation}}

\hypertarget{check-the-structure-of-the-data-frame}{%
\subsubsection{Check the structure of the data
frame}\label{check-the-structure-of-the-data-frame}}

We can see that the data is mix of Int and Chars.

\begin{Shaded}
\begin{Highlighting}[]
\FunctionTok{str}\NormalTok{(crash\_data)}
\end{Highlighting}
\end{Shaded}

\begin{verbatim}
## 'data.frame':    1000 obs. of  10 variables:
##  $ Crash_Year               : int  2013 2013 2013 2013 2013 2013 2013 2013 2013 2013 ...
##  $ Crash_Month              : chr  "January" "January" "January" "January" ...
##  $ County                   : chr  "Barry" "Barry" "Bay" "Berrien" ...
##  $ Crash_Type               : chr  "Head-On" "Rear-End" "Single Motor Vehicle" "Rear-End" ...
##  $ Crash_Drug_Use           : chr  "No Drugs Involved" "No Drugs Involved" "No Drugs Involved" "No Drugs Involved" ...
##  $ Crash_Drinking           : chr  "No Drinking Involved" "No Drinking Involved" "No Drinking Involved" "No Drinking Involved" ...
##  $ Crash_Hit_and_Run        : chr  "Not Hit-and-Run" "Not Hit-and-Run" "Not Hit-and-Run" "Not Hit-and-Run" ...
##  $ Speed_Limit_at_Crash_Site: int  55 55 55 70 55 45 55 55 55 55 ...
##  $ Total_Motor_Vehicles     : int  2 2 1 2 2 2 1 2 2 1 ...
##  $ Crash                    : int  0 0 0 0 0 0 0 0 0 0 ...
\end{verbatim}

\hypertarget{checking-na-values}{%
\subsubsection{checking NA values}\label{checking-na-values}}

There are 6 NA values in Speed Limit column, So imputing with mean.

\begin{Shaded}
\begin{Highlighting}[]
\FunctionTok{sum}\NormalTok{(}\FunctionTok{is.na}\NormalTok{(crash\_data))}
\end{Highlighting}
\end{Shaded}

\begin{verbatim}
## [1] 6
\end{verbatim}

\begin{Shaded}
\begin{Highlighting}[]
\ControlFlowTok{for}\NormalTok{(i }\ControlFlowTok{in} \DecValTok{1}\SpecialCharTok{:}\FunctionTok{ncol}\NormalTok{(crash\_data))\{}
\NormalTok{  crash\_data[}\FunctionTok{is.na}\NormalTok{(crash\_data[,i]), i] }\OtherTok{\textless{}{-}} \FunctionTok{mean}\NormalTok{(crash\_data[,i], }\AttributeTok{na.rm =} \ConstantTok{TRUE}\NormalTok{)}
\NormalTok{\}}
\end{Highlighting}
\end{Shaded}

\begin{verbatim}
## Warning in mean.default(crash_data[, i], na.rm = TRUE): argument is not numeric
## or logical: returning NA

## Warning in mean.default(crash_data[, i], na.rm = TRUE): argument is not numeric
## or logical: returning NA

## Warning in mean.default(crash_data[, i], na.rm = TRUE): argument is not numeric
## or logical: returning NA

## Warning in mean.default(crash_data[, i], na.rm = TRUE): argument is not numeric
## or logical: returning NA

## Warning in mean.default(crash_data[, i], na.rm = TRUE): argument is not numeric
## or logical: returning NA

## Warning in mean.default(crash_data[, i], na.rm = TRUE): argument is not numeric
## or logical: returning NA
\end{verbatim}

\hypertarget{create-a-bar-chart-of-the-number-of-crashes-by-county}{%
\subsubsection{Create a bar chart of the number of crashes by
county}\label{create-a-bar-chart-of-the-number-of-crashes-by-county}}

We can see that Wayne county has highest number of crashes.

\begin{Shaded}
\begin{Highlighting}[]
\NormalTok{crash\_count\_by\_county }\OtherTok{\textless{}{-}}\NormalTok{ crash\_data }\SpecialCharTok{\%\textgreater{}\%}
  \FunctionTok{group\_by}\NormalTok{(County) }\SpecialCharTok{\%\textgreater{}\%}
  \FunctionTok{summarise}\NormalTok{(}\AttributeTok{count =} \FunctionTok{n}\NormalTok{()) }\SpecialCharTok{\%\textgreater{}\%}
  \FunctionTok{arrange}\NormalTok{(}\FunctionTok{desc}\NormalTok{(count)) }\SpecialCharTok{\%\textgreater{}\%}
  \FunctionTok{head}\NormalTok{(}\DecValTok{10}\NormalTok{)}

\FunctionTok{ggplot}\NormalTok{(crash\_count\_by\_county, }\FunctionTok{aes}\NormalTok{(}\AttributeTok{x =}\NormalTok{ County, }\AttributeTok{y =}\NormalTok{ count, }\AttributeTok{fill =}\NormalTok{ County)) }\SpecialCharTok{+}
  \FunctionTok{geom\_bar}\NormalTok{(}\AttributeTok{stat =} \StringTok{"identity"}\NormalTok{) }\SpecialCharTok{+}
  \FunctionTok{labs}\NormalTok{(}\AttributeTok{title =} \StringTok{"Top 10 Counties by Number of Crashes"}\NormalTok{, }\AttributeTok{x =} \StringTok{"County"}\NormalTok{, }\AttributeTok{y =} \StringTok{"Number of Crashes"}\NormalTok{) }\SpecialCharTok{+}
  \FunctionTok{theme}\NormalTok{(}\AttributeTok{axis.text.x =} \FunctionTok{element\_text}\NormalTok{(}\AttributeTok{angle =} \DecValTok{45}\NormalTok{, }\AttributeTok{hjust =} \DecValTok{1}\NormalTok{))}
\end{Highlighting}
\end{Shaded}

\includegraphics{Crash_Data_files/figure-latex/unnamed-chunk-5-1.pdf}

\hypertarget{create-a-scatter-plot-of-speed-limit-vs-number-of-vehicles-involved}{%
\subsubsection{Create a scatter plot of speed limit vs number of
vehicles
involved}\label{create-a-scatter-plot-of-speed-limit-vs-number-of-vehicles-involved}}

It has been observed that As the Speed Limit increases the vehicle
including in crash increases.

\begin{Shaded}
\begin{Highlighting}[]
\FunctionTok{ggplot}\NormalTok{(crash\_data, }\FunctionTok{aes}\NormalTok{(}\AttributeTok{x =}\NormalTok{ Speed\_Limit\_at\_Crash\_Site, }\AttributeTok{y =}\NormalTok{ Total\_Motor\_Vehicles)) }\SpecialCharTok{+}
  \FunctionTok{geom\_point}\NormalTok{(}\AttributeTok{alpha =} \FloatTok{0.5}\NormalTok{) }\SpecialCharTok{+}
  \FunctionTok{labs}\NormalTok{(}\AttributeTok{title =} \StringTok{"Speed Limit vs Number of Vehicles Involved in Crashes"}\NormalTok{, }\AttributeTok{x =} \StringTok{"Speed Limit at Crash Site"}\NormalTok{, }\AttributeTok{y =} \StringTok{"Number of Vehicles Involved"}\NormalTok{)}
\end{Highlighting}
\end{Shaded}

\includegraphics{Crash_Data_files/figure-latex/unnamed-chunk-6-1.pdf}

\hypertarget{create-a-stacked-bar-chart-of-the-number-of-crashes-by-crash-type-and-drug-use}{%
\subsubsection{Create a stacked bar chart of the number of crashes by
crash type and drug
use}\label{create-a-stacked-bar-chart-of-the-number-of-crashes-by-crash-type-and-drug-use}}

\begin{itemize}
\tightlist
\item
  People Driving single has more number of crashes.
\item
  Head on and angle has the similar number of crashes.
\end{itemize}

\begin{Shaded}
\begin{Highlighting}[]
\NormalTok{crash\_count\_by\_type\_and\_drug\_use }\OtherTok{\textless{}{-}}\NormalTok{ crash\_data }\SpecialCharTok{\%\textgreater{}\%}
  \FunctionTok{group\_by}\NormalTok{(Crash\_Type, Crash\_Drug\_Use) }\SpecialCharTok{\%\textgreater{}\%}
  \FunctionTok{summarise}\NormalTok{(}\AttributeTok{count =} \FunctionTok{n}\NormalTok{()) }\SpecialCharTok{\%\textgreater{}\%}
  \FunctionTok{arrange}\NormalTok{(}\FunctionTok{desc}\NormalTok{(count))}
\end{Highlighting}
\end{Shaded}

\begin{verbatim}
## `summarise()` has grouped output by 'Crash_Type'. You can override using the
## `.groups` argument.
\end{verbatim}

\begin{Shaded}
\begin{Highlighting}[]
\FunctionTok{ggplot}\NormalTok{(crash\_count\_by\_type\_and\_drug\_use, }\FunctionTok{aes}\NormalTok{(}\AttributeTok{x =}\NormalTok{ Crash\_Type, }\AttributeTok{y =}\NormalTok{ count, }\AttributeTok{fill =}\NormalTok{ Crash\_Drug\_Use)) }\SpecialCharTok{+}
  \FunctionTok{geom\_bar}\NormalTok{(}\AttributeTok{stat =} \StringTok{"identity"}\NormalTok{) }\SpecialCharTok{+}
  \FunctionTok{labs}\NormalTok{(}\AttributeTok{title =} \StringTok{"Number of Crashes by Crash Type and Drug Use"}\NormalTok{, }\AttributeTok{x =} \StringTok{"Crash Type"}\NormalTok{, }\AttributeTok{y =} \StringTok{"Number of Crashes"}\NormalTok{) }\SpecialCharTok{+}
  \FunctionTok{theme}\NormalTok{(}\AttributeTok{axis.text.x =} \FunctionTok{element\_text}\NormalTok{(}\AttributeTok{angle =} \DecValTok{45}\NormalTok{, }\AttributeTok{hjust =} \DecValTok{1}\NormalTok{))}
\end{Highlighting}
\end{Shaded}

\includegraphics{Crash_Data_files/figure-latex/unnamed-chunk-7-1.pdf}

\hypertarget{heatmap-of-crashes-by-month-and-county}{%
\subsubsection{Heatmap of crashes by month and
county}\label{heatmap-of-crashes-by-month-and-county}}

It is difficult to find info from Heatmap. But We can see that Wayne
county has highest number of crashes and most are between Jan to Feb.

\begin{Shaded}
\begin{Highlighting}[]
\NormalTok{crash\_count\_by\_month\_and\_county }\OtherTok{\textless{}{-}}\NormalTok{ crash\_data }\SpecialCharTok{\%\textgreater{}\%}
  \FunctionTok{group\_by}\NormalTok{(Crash\_Month, County) }\SpecialCharTok{\%\textgreater{}\%}
  \FunctionTok{summarise}\NormalTok{(}\AttributeTok{count =} \FunctionTok{n}\NormalTok{()) }\SpecialCharTok{\%\textgreater{}\%}
  \FunctionTok{arrange}\NormalTok{(Crash\_Month, County)}
\end{Highlighting}
\end{Shaded}

\begin{verbatim}
## `summarise()` has grouped output by 'Crash_Month'. You can override using the
## `.groups` argument.
\end{verbatim}

\begin{Shaded}
\begin{Highlighting}[]
\FunctionTok{ggplot}\NormalTok{(crash\_count\_by\_month\_and\_county, }\FunctionTok{aes}\NormalTok{(}\AttributeTok{x =}\NormalTok{ County, }\AttributeTok{y =}\NormalTok{ Crash\_Month, }\AttributeTok{fill =}\NormalTok{ count)) }\SpecialCharTok{+}
  \FunctionTok{geom\_tile}\NormalTok{() }\SpecialCharTok{+}
  \FunctionTok{scale\_fill\_gradient}\NormalTok{(}\AttributeTok{low =} \StringTok{"white"}\NormalTok{, }\AttributeTok{high =} \StringTok{"red"}\NormalTok{) }\SpecialCharTok{+}
  \FunctionTok{labs}\NormalTok{(}\AttributeTok{title =} \StringTok{"Heatmap of Crashes by Month and County"}\NormalTok{, }\AttributeTok{x =} \StringTok{"County"}\NormalTok{, }\AttributeTok{y =} \StringTok{"Month"}\NormalTok{) }\SpecialCharTok{+}
  \FunctionTok{theme}\NormalTok{(}\AttributeTok{axis.text.x =} \FunctionTok{element\_text}\NormalTok{(}\AttributeTok{angle =} \DecValTok{45}\NormalTok{, }\AttributeTok{hjust =} \DecValTok{1}\NormalTok{))}
\end{Highlighting}
\end{Shaded}

\includegraphics{Crash_Data_files/figure-latex/unnamed-chunk-8-1.pdf}

\hypertarget{box-plot-of-speed-limit-by-crash-type}{%
\subsubsection{Box plot of speed limit by crash
type}\label{box-plot-of-speed-limit-by-crash-type}}

We can see some type has outliers but most of the medians are at the
upper level, this may be because most of the accidents happens at high
speed and most with sideswipe and angle.

\begin{Shaded}
\begin{Highlighting}[]
\FunctionTok{ggplot}\NormalTok{(crash\_data, }\FunctionTok{aes}\NormalTok{(}\AttributeTok{x =}\NormalTok{ Crash\_Type, }\AttributeTok{y =}\NormalTok{ Speed\_Limit\_at\_Crash\_Site)) }\SpecialCharTok{+}
  \FunctionTok{geom\_boxplot}\NormalTok{() }\SpecialCharTok{+}
  \FunctionTok{labs}\NormalTok{(}\AttributeTok{title =} \StringTok{"Box Plot of Speed Limit by Crash Type"}\NormalTok{, }\AttributeTok{x =} \StringTok{"Crash Type"}\NormalTok{, }\AttributeTok{y =} \StringTok{"Speed Limit at Crash Site"}\NormalTok{) }\SpecialCharTok{+}
  \FunctionTok{theme}\NormalTok{(}\AttributeTok{axis.text.x =} \FunctionTok{element\_text}\NormalTok{(}\AttributeTok{angle =} \DecValTok{45}\NormalTok{, }\AttributeTok{hjust =} \DecValTok{1}\NormalTok{))}
\end{Highlighting}
\end{Shaded}

\includegraphics{Crash_Data_files/figure-latex/unnamed-chunk-9-1.pdf}

\hypertarget{stacked-bar-chart-of-crashes-by-crash-type-and-drinking-involvement}{%
\subsubsection{Stacked bar chart of crashes by crash type and drinking
involvement}\label{stacked-bar-chart-of-crashes-by-crash-type-and-drinking-involvement}}

\begin{itemize}
\tightlist
\item
  Please Driving single has more number of crashes.
\item
  Head on and angle has the similar number of crashes.
\end{itemize}

\begin{Shaded}
\begin{Highlighting}[]
\NormalTok{crash\_count\_by\_type\_and\_drinking }\OtherTok{\textless{}{-}}\NormalTok{ crash\_data }\SpecialCharTok{\%\textgreater{}\%}
  \FunctionTok{group\_by}\NormalTok{(Crash\_Type, Crash\_Drinking) }\SpecialCharTok{\%\textgreater{}\%}
  \FunctionTok{summarise}\NormalTok{(}\AttributeTok{count =} \FunctionTok{n}\NormalTok{()) }\SpecialCharTok{\%\textgreater{}\%}
  \FunctionTok{arrange}\NormalTok{(}\FunctionTok{desc}\NormalTok{(count))}
\end{Highlighting}
\end{Shaded}

\begin{verbatim}
## `summarise()` has grouped output by 'Crash_Type'. You can override using the
## `.groups` argument.
\end{verbatim}

\begin{Shaded}
\begin{Highlighting}[]
\FunctionTok{ggplot}\NormalTok{(crash\_count\_by\_type\_and\_drinking, }\FunctionTok{aes}\NormalTok{(}\AttributeTok{x =}\NormalTok{ Crash\_Type, }\AttributeTok{y =}\NormalTok{ count, }\AttributeTok{fill =}\NormalTok{ Crash\_Drinking)) }\SpecialCharTok{+}
  \FunctionTok{geom\_bar}\NormalTok{(}\AttributeTok{stat =} \StringTok{"identity"}\NormalTok{) }\SpecialCharTok{+}
  \FunctionTok{labs}\NormalTok{(}\AttributeTok{title =} \StringTok{"Number of Crashes by Crash Type and Drinking Involvement"}\NormalTok{, }\AttributeTok{x =} \StringTok{"Crash Type"}\NormalTok{, }\AttributeTok{y =} \StringTok{"Number of Crashes"}\NormalTok{) }\SpecialCharTok{+}
  \FunctionTok{theme}\NormalTok{(}\AttributeTok{axis.text.x =} \FunctionTok{element\_text}\NormalTok{(}\AttributeTok{angle =} \DecValTok{45}\NormalTok{, }\AttributeTok{hjust =} \DecValTok{1}\NormalTok{))}
\end{Highlighting}
\end{Shaded}

\includegraphics{Crash_Data_files/figure-latex/unnamed-chunk-10-1.pdf}

\hypertarget{statistical-analysis}{%
\section{Statistical Analysis}\label{statistical-analysis}}

\hypertarget{hypothesis-test-for-the-difference-in-means-of-speed-limit-between-injury-and-fatal-crashes}{%
\subsubsection{Hypothesis test for the difference in means of speed
limit between injury and fatal
crashes:}\label{hypothesis-test-for-the-difference-in-means-of-speed-limit-between-injury-and-fatal-crashes}}

\begin{Shaded}
\begin{Highlighting}[]
\CommentTok{\# Subset data for injury and fatal crashes}
\NormalTok{injury\_data }\OtherTok{\textless{}{-}} \FunctionTok{subset}\NormalTok{(crash\_data, Crash }\SpecialCharTok{==} \DecValTok{0}\NormalTok{)}
\NormalTok{fatal\_data }\OtherTok{\textless{}{-}} \FunctionTok{subset}\NormalTok{(crash\_data, Crash }\SpecialCharTok{==} \DecValTok{1}\NormalTok{)}

\CommentTok{\# Conduct t{-}test for difference in means}
\NormalTok{t\_test\_result }\OtherTok{\textless{}{-}} \FunctionTok{t.test}\NormalTok{(injury\_data}\SpecialCharTok{$}\NormalTok{Speed\_Limit\_at\_Crash\_Site, fatal\_data}\SpecialCharTok{$}\NormalTok{Speed\_Limit\_at\_Crash\_Site)}

\CommentTok{\# Print results}
\FunctionTok{cat}\NormalTok{(}\StringTok{"t{-}test for difference in means of speed limit between injury and fatal crashes:}\SpecialCharTok{\textbackslash{}n}\StringTok{"}\NormalTok{)}
\end{Highlighting}
\end{Shaded}

\begin{verbatim}
## t-test for difference in means of speed limit between injury and fatal crashes:
\end{verbatim}

\begin{Shaded}
\begin{Highlighting}[]
\FunctionTok{cat}\NormalTok{(}\StringTok{"t{-}value:"}\NormalTok{, t\_test\_result}\SpecialCharTok{$}\NormalTok{statistic, }\StringTok{"}\SpecialCharTok{\textbackslash{}n}\StringTok{"}\NormalTok{)}
\end{Highlighting}
\end{Shaded}

\begin{verbatim}
## t-value: 0.3511422
\end{verbatim}

\begin{Shaded}
\begin{Highlighting}[]
\FunctionTok{cat}\NormalTok{(}\StringTok{"p{-}value:"}\NormalTok{, t\_test\_result}\SpecialCharTok{$}\NormalTok{p.value, }\StringTok{"}\SpecialCharTok{\textbackslash{}n}\StringTok{"}\NormalTok{)}
\end{Highlighting}
\end{Shaded}

\begin{verbatim}
## p-value: 0.7281002
\end{verbatim}

The t-test results show the t-value and p-value for the difference in
means of speed limit between injury and fatal crashes. The t-value
represents the difference in means divided by the standard error of the
difference. The p-value represents the probability of observing a
t-value as extreme or more extreme than the observed t-value, assuming
the null hypothesis that the means are equal. A small p-value (less than
the significance level, typically 0.05) indicates strong evidence
against the null hypothesis and in favor of the alternative hypothesis
that the means are different. In this case, if the p-value is large, we
can conclude that there is NO significant difference in the mean speed
limit between injury and fatal crashes.

\hypertarget{logistic-regression-to-predict-the-likelihood-of-a-fatal-crash-based-on-crash-type-and-drinking-involvement}{%
\subsubsection{Logistic regression to predict the likelihood of a fatal
crash based on crash type and drinking
involvement:}\label{logistic-regression-to-predict-the-likelihood-of-a-fatal-crash-based-on-crash-type-and-drinking-involvement}}

\begin{Shaded}
\begin{Highlighting}[]
\CommentTok{\# Create binary variables for crash type and drinking involvement}
\NormalTok{crash\_data}\SpecialCharTok{$}\NormalTok{Crash\_Type\_Binary }\OtherTok{\textless{}{-}} \FunctionTok{ifelse}\NormalTok{(crash\_data}\SpecialCharTok{$}\NormalTok{Crash\_Type }\SpecialCharTok{\%in\%} \FunctionTok{c}\NormalTok{(}\StringTok{"Head{-}On"}\NormalTok{, }\StringTok{"Rear{-}End"}\NormalTok{), }\DecValTok{0}\NormalTok{, }\DecValTok{1}\NormalTok{)}
\NormalTok{crash\_data}\SpecialCharTok{$}\NormalTok{Drinking\_Binary }\OtherTok{\textless{}{-}} \FunctionTok{ifelse}\NormalTok{(crash\_data}\SpecialCharTok{$}\NormalTok{Crash\_Drinking }\SpecialCharTok{==} \StringTok{"Drinking Involved"}\NormalTok{, }\DecValTok{1}\NormalTok{, }\DecValTok{0}\NormalTok{)}

\CommentTok{\# Fit logistic regression model}
\NormalTok{log\_reg\_model }\OtherTok{\textless{}{-}} \FunctionTok{glm}\NormalTok{(Crash }\SpecialCharTok{\textasciitilde{}}\NormalTok{ Crash\_Type\_Binary }\SpecialCharTok{+}\NormalTok{ Drinking\_Binary, }\AttributeTok{data =}\NormalTok{ crash\_data, }\AttributeTok{family =} \StringTok{"binomial"}\NormalTok{)}

\CommentTok{\# Print model summary}
\FunctionTok{summary}\NormalTok{(log\_reg\_model)}
\end{Highlighting}
\end{Shaded}

\begin{verbatim}
## 
## Call:
## glm(formula = Crash ~ Crash_Type_Binary + Drinking_Binary, family = "binomial", 
##     data = crash_data)
## 
## Deviance Residuals: 
##     Min       1Q   Median       3Q      Max  
## -0.2828  -0.2828  -0.2198  -0.1671   3.0933  
## 
## Coefficients:
##                   Estimate Std. Error z value Pr(>|z|)    
## (Intercept)        -3.7110     0.4565  -8.130  4.3e-16 ***
## Crash_Type_Binary   0.5118     0.5001   1.024   0.3060    
## Drinking_Binary    -1.0649     0.5451  -1.954   0.0507 .  
## ---
## Signif. codes:  0 '***' 0.001 '**' 0.01 '*' 0.05 '.' 0.1 ' ' 1
## 
## (Dispersion parameter for binomial family taken to be 1)
## 
##     Null deviance: 255.44  on 999  degrees of freedom
## Residual deviance: 249.74  on 997  degrees of freedom
## AIC: 255.74
## 
## Number of Fisher Scoring iterations: 7
\end{verbatim}

The logistic regression model results show the coefficients, standard
errors, z-values, and p-values for the binary variables for crash type
and drinking involvement. The coefficients represent the log-odds of a
fatal crash associated with each variable, holding all other variables
constant. The standard errors represent the uncertainty in the
coefficient estimates. The z-values are the coefficients divided by
their standard errors, which are used to calculate the p-values. A small
p-value (less than the significance level, typically 0.05) for a
coefficient indicates strong evidence that the variable is associated
with the outcome. In this case, if the p-values for the crash type and
drinking involvement variables are small, we can conclude that these
variables are significant predictors of the likelihood of a fatal crash.

\hypertarget{anova-for-the-effect-of-county-on-the-number-of-vehicles-involved-in-a-crash}{%
\subsubsection{ANOVA for the effect of county on the number of vehicles
involved in a
crash:}\label{anova-for-the-effect-of-county-on-the-number-of-vehicles-involved-in-a-crash}}

\begin{Shaded}
\begin{Highlighting}[]
\CommentTok{\# Fit ANOVA model}
\NormalTok{anova\_result }\OtherTok{\textless{}{-}} \FunctionTok{aov}\NormalTok{(Total\_Motor\_Vehicles }\SpecialCharTok{\textasciitilde{}}\NormalTok{ County, }\AttributeTok{data =}\NormalTok{ crash\_data)}

\CommentTok{\# Print results}
\FunctionTok{cat}\NormalTok{(}\StringTok{"ANOVA for effect of county on number of vehicles involved in a crash:}\SpecialCharTok{\textbackslash{}n}\StringTok{"}\NormalTok{)}
\end{Highlighting}
\end{Shaded}

\begin{verbatim}
## ANOVA for effect of county on number of vehicles involved in a crash:
\end{verbatim}

\begin{Shaded}
\begin{Highlighting}[]
\FunctionTok{summary}\NormalTok{(anova\_result)}
\end{Highlighting}
\end{Shaded}

\begin{verbatim}
##              Df Sum Sq Mean Sq F value Pr(>F)
## County       77    172   2.230   0.501      1
## Residuals   922   4101   4.448
\end{verbatim}

The ANOVA results show the sum of squares, degrees of freedom, mean
squares, F-value, and p-value for the effect of county on the number of
vehicles involved in a crash. The sum of squares represents the
variation explained by the county variable, and the degrees of freedom
represent the number of categories minus one. The mean squares are the
sum of squares divided by the degrees of freedom. The F-value is the
mean square for county divided by the mean square for error (unexplained
variation), which is used to calculate the p-value. A small p-value
(less than the significance level, typically 0.05) indicates strong
evidence against the null hypothesis that the means are equal and in
favor of the alternative hypothesis that there is a significant effect
of county on the number of vehicles involved in a crash. In this case,
if the p-value is large, we can conclude that there is No significant
effect of county on the number of vehicles involved in a crash.

\end{document}
